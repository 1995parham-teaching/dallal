\documentclass[]{article}

\usepackage{hyperref}
\usepackage{import}
\usepackage{listings}
\usepackage{minted}
\usepackage[dvipsnames]{xcolor}
\usepackage{hyperref}
\usepackage{graphicx}
\usepackage{geometry}
 \geometry{
 a4paper,
 left=2.2cm,
 right=2.2cm,
 top=2.5cm,
 bottom=2.5cm
 }

\usepackage[localise]{xepersian}
\settextfont{Vazir}
\setlatintextfont{Roboto}

% for listing environments
\colorlet{LightGray}{Gray!30!}

% for href
\eqcommand{تارنما}{href}

\عنوان{دلال پیام - قسمت دوم}
\نویسنده{پرهام الوانی}
\تاریخ{۴ دی ۱۴۰۰}

\begin{document}
  \عنوان‌ساز
  \فهرست‌مطالب
  \صفحه‌شکن

  \قسمت{مقدمه}
  در قسمت اول پروژه با دلال پیام آشنا شدید. در این قسمت قصد داریم این سامانه را در شرایط مختلف
  و به منظور یادگیری مفاهیم کنترل جریان و ازدحام بررسی کنیم.

  \قسمت{کنترل جریان با \متن‌لاتین{TCP}}

  سیستم دلال پیام را همانطور که از قسمت پیشین توسعه داده‌اید، اجرا کنید، یک کلاینت بنوسید که روی یک تاپیک خاص مشترک شده
  و ارتباط \متن‌لاتین{TCP} خود را باز نگهدارد، تا به اینجا این کلاینت دقیقا مشابه با کلاینتی است که پیشتر برای اشتراک روی یک تاپیک توسعه داده بودید،
  اما این کلاینت هیچ داده‌ای را از روی ارتباط \متن‌لاتین{TCP} \متن‌سیاه{نمی‌خواند}.
  کلاینت دیگری اجرا کنید که روی این تاپیک خاص شروع به انتشار اطلاعات کند، این انتشار می‌بایست به صورت تکراری ادامه پیدا کند.
  در این شرایط این ارتباط را در نرم‌افزار \متن‌لاتین{Wireshark} بررسی کنید، بعد از گذشت چه مدت زمان داده‌ای منتقل نمی‌شود؟
  آیا دلال پیام شما در این شرایط خطا می‌دهد؟

  \قسمت{بیشترین تعداد ارتباط همزمان \متن‌لاتین{TCP}}

  دلال پیام را همانطور که از قسمت پیشین توسعه داده‌اید، اجرا کنید. همانطور که پیش اشاره شد، کلاینت‌هایی که مشترک می‌شدند
  یک ارتباط باز با این دلال پیام را نگهداری می‌کردند. تعداد این کلاینتها را افزایش دهید، در نظر داشته باشید که نیازی به کلاینت برای انتشار
  پیام نیست و تنها افزایش تعداد کلاینت‌های متشرک شونده کفایت می‌کند.
  تا چه میزان کلاینت می‌توانید داشته باشید؟ آیا این تعداد با حداکثر تعداد
  پورت‌های موچود در ارتباط \متن‌لاتین{TCP} قابل مقایسه است؟ چه عاملی باعث این محدودیت می‌شود؟

  \قسمت{پیاده‌سازی دلال پیام با \متن‌لاتین{UDP}}

  قصد داریم دلال پیامی که پبیشتر نوشته‌ایم، در کنار پشتیبانی از ارتباط \متن‌لاتین{TCP}، از ارتباط \متن‌لاتین{UDP} هم پشتیبانی کند.
  تفاوت اصلی در ارتباط \متن‌لاتین{UDP} نبود یک ارتباط پایدار میان سرور و کلاینت است و این مساله در زمان مشترک شدن
  کلاینتها روی تاپیک‌ها بیشتر خود را نشان می‌دهد چرا که سرور می‌بایست پیام‌ها را برای مشترکین فعال ارسال کند.

  \زیرقسمت{اطمینان از سلامت طرف مقابل}
  در این حالت پروتکل مشابه آنچه پیشتر آورده شد باقی می‌ماند اما برای پیدا کردن مشترکین فعال نیاز داریم تا از پیام‌های
  \متن‌لاتین{Ping} و \متن‌لاتین{Pong} استفاده کنیم. این پیام‌ها به سرور و کلاینت اجازه می‌دهند بر پایه ارتباط \متن‌لاتین{UDP}
  از فعال بودن طرف مقابل اطمینان حاصل کنند.

  بنابراین پیاده‌سازی پیام‌های \متن‌لاتین{Ping} و \متن‌لاتین{Pong} در این پیاده‌سازی پروتکل \متن‌لاتین{UDP} اجباری هستند
  و شما از این طریق می‌توانید کلاینتهایی که ارتباطشان قطع شده است را تشخیص دهید. برای این امر هر طرف در دوره‌های ۱۰
  ثانیه‌ای پیام‌های \متن‌لاتین{Ping} را ارسال می‌کند و اگر تا شروع دوره‌ی بعدی پاسخی دریافت نکند به منزله خطا در اتصال خواهد بود.

  \زیرقسمت{ذخیره کردن اطلاعات طرف مقابل}

  کلاینت‌های \متن‌لاتین{UDP} برای ارتباط با سرور تنها نیاز به پورت و آدرس دلال پیام دارند، اما سرور برای ارسال پیام به کلاینت‌ها
  می‌بایست از آخرین پورت و آی پی استفاده کند که با آن از کلاینت پیام دریافت کرده است. پس شما می‌بایست در هنگام توسعه
  کلاینت با \متن‌لاتین{UDP} به این نکته توجه داشته باشید ابتدا روی یک پورت مشخص \متن‌لاتین{Bind}
  کنید تا سرور بتواند شما را پیدا کند.

  \begin{latin}
  \begin{minted}[bgcolor=LightGray]{python}
import socket

client = socket.socket(socket.AF_INET, socket.SOCK_DGRAM) # UDP
client.bind(("", 37020))
while True:
    data, addr = client.recvfrom(1024)
    print("received message:",data,addr)
  \end{minted}
  \end{latin}

  در کد نمونه آورده شده سوکت \متن‌لاتین{UDP} به پورت ۳۷۰۲۰ بسته شده است و به این ترتیب بسته‌های ارسال از پورت مبدا ۳۷۰۲۰ ارسال شده
  و از سوی دیگر می‌توان روی این پورت مطابق آنچه در مثال آورده شده است بسته دریافت نمود.
  \فضای‌و*{\پر}
  \شروع{وسط‌چین}
این سند برپایه بسته \متن‌لاتین{\زی‌پرشین} گونه \متن‌لاتین{\گونه‌زی‌پرشین} توسعه پیدا کرده است.
  \پایان{وسط‌چین}

\end{document}
