\documentclass[]{article}

\usepackage{hyperref}
\usepackage{import}
\usepackage{listings}
\usepackage{minted}
\usepackage[dvipsnames]{xcolor}
\usepackage{hyperref}
\usepackage{graphicx}
\usepackage{geometry}
 \geometry{
 a4paper,
 left=2.2cm,
 right=2.2cm,
 top=2.5cm,
 bottom=2.5cm
 }

\usepackage[localise]{xepersian}
\settextfont{Vazir}
\setlatintextfont{Roboto}

% for listing environments
\colorlet{LightGray}{Gray!30!}

% for href
\eqcommand{تارنما}{href}

\عنوان{دلال پیام - قسمت دوم}
\نویسنده{پرهام الوانی}
\تاریخ{۴ دی ۱۴۰۰}

\begin{document}
  \عنوان‌ساز
  \فهرست‌مطالب
  \صفحه‌شکن

  \قسمت{مقدمه}

  در قسمت اول پروژه با دلال پیام آشنا شدید. در این قسمت قصد داریم این سامانه را در شرایط مختلف
  و به منظور یادگیری مفاهیم کنترل جریان و ازدحام بررسی کنیم.

  \قسمت{کنترل جریان با \متن‌لاتین{TCP}}

  \شروع{شمارش}
  \فقره یک نسخه از برنامه ای که در پروژه قبل نوشته اید را در حالت کلاینت \متن‌لاتین{subscriber} برای موضوع \متن‌لاتین{flow-control-tcp} اجرا کنید.
  \فقره نسخه دیگری از برنامه را در حالت کلاینت \متن‌لاتین{publish} اجرا کنید و به صورت تناوبی (در حلقه) پیام‌هایی را توسط آن داخل سابجکت \متن‌لاتین{flow-control-tcp} بریزید.
  \فقره وضعیت بسته های شبکه در فرایند بالا را با استفاده از وایرشارک بررسی و تفسیر کنید
  \پایان{شمارش}

  \قسمت{بیشترین تعداد ارتباط همزمان \متن‌لاتین{TCP}}

  دلال پیام را همانطور که از قسمت پیشین توسعه داده‌اید، اجرا کنید. همانطور که پیش اشاره شد، کلاینت‌هایی که مشترک می‌شدند
  یک ارتباط باز با این دلال پیام را نگهداری می‌کردند.
  کلاینت برنامه را در حالت \متن‌لاتین{subscriber} چندین بار (بهتر است اجرا و شروع برنامه را در قالب یک حلقه انجام دهید) اجرا کنید.
  حد بالای تعداد کلاینت‌های درحال اجرا چه تعداد است؟
  آیا این تعداد با حداکثر تعداد
  پورت‌های موچود در ارتباط \متن‌لاتین{TCP} قابل مقایسه است؟ چه عاملی باعث این محدودیت می‌شود؟

  \قسمت{پیاده‌سازی دلال پیام با \متن‌لاتین{UDP}}

  قصد داریم دلال پیامی که پبیشتر نوشته‌ایم، در کنار پشتیبانی از ارتباط \متن‌لاتین{TCP}، از ارتباط \متن‌لاتین{UDP} هم پشتیبانی کند.
  تفاوت اصلی در ارتباط \متن‌لاتین{UDP} نبود یک ارتباط پایدار میان سرور و کلاینت است و این مساله در زمان مشترک شدن
  کلاینتها روی تاپیک‌ها بیشتر خود را نشان می‌دهد چرا که سرور می‌بایست پیام‌ها را برای مشترکین فعال ارسال کند.
  در ادامه به این موضوعات پرداخته می‌شود.


  \زیرقسمت{‌پیاده‌سازی دلال پیام با استفاده از \متن‌لاتین{UDP}}

  مانند قبل سرور می‌بایست بر روی یک پورت و آدرس مشخص قابل دسترسی باشد از این رو سوکتی که برای سرور در نظر گرفته می‌شود
  با دستور \متن‌لاتین{Bind} روی یک آدرس و پورت مشخص بسته می‌شود.

  در قسمت قبلی پیاده‌سازی \متن‌لاتین{TCP} برای مدیریت بسته‌های رسیده به هر سوکت از یک \متن‌لاتین{Thread} استفاده شد.
  در این حالت می‌توان دریافت بسته ها را همگی در یک \متن‌لاتین{Thread} انجام داد و مدیریت و کنترل جریان پیام ها را در سرور دلال پیام انجام داد.

  برنامه شما در پورت ۱۲۳۴ \متن‌لاتین{UDP} اقدام به شنود بسته‌ها می‌کند.
  (برخلاف ارتباط \متن‌لاتین{TCP} که هر کلاینت با ایجاد یک سوکت اختصاصی بسته‌های خود را برای سرور ارسال می‌کند و سرور باید رسیدن بسته ها را در ترد مخصوص سوکت ایجاد شده پردازش کند).
  شما می‌توانید به محض رسیدن بسته‌ها از هر کلاینتی آن را در سرور خود دریافت کنید.
  پردازش مبدا بسته‌ها با استفاده از هدرهای بسته‌های \متن‌لاتین{UDP} انجام می‌شود و باید از اطلاعات آن جهت مدیریت کلاینت‌ها استفاده کنید.

  \begin{latin}
  \begin{minted}[bgcolor=LightGray]{python}
import socket

client = socket.socket(socket.AF_INET, socket.SOCK_DGRAM) # UDP
client.bind(('', 1234))
while True:
    data, addr = client.recvfrom(1024)
    print("received message:",data,addr)
  \end{minted}
  \end{latin}

  \زیرقسمت{اطمینان از سلامت طرف مقابل}

  از آنجایی که در پروتکل \متن‌لاتین{UDP} مکانیزمی جهت تشخیص برقراری یا قطع ارتباط در لایه ارتباط (\متن‌لاتین{transport}) وجود ندارد.
  از شما خواسته شده که این امکان را در لایه های بالاتر (\متن‌لاتین{application}) فراهم کنید.
  به این منظور کلاینت پس از ارسال اولین بسته به سرور (معادل برقراری کانکشن در ارتباط \متن‌لاتین{TCP})
  به صورت تناوبی (در یک حلقه) اقدام به چک کردن سرور در قالب پیام‌های \متن‌لاتین{ping/pong}ای که در فاز قبلی تعریف کردید می‌کند.

  برای این کار لازم است که در \متن‌سیاه{کلاینت} به صورت تناوبی (هر ۱۰ ثانیه یکبار) یک بسته \متن‌لاتین{ping} برای سرور ارسال کنید.
  سالم بودن اتصال برای کلاینت به معنای دریافت \متن‌لاتین{pong} بعد از ارسال \متن‌لاتین{ping} است.

  سمت سرور باید بعد از شروع یک اتصال (خواندن یک پیام از آدرس جدید در سوکت) اقدام به چک کردن پیام‌های \متن‌لاتین{ping} کلاینت کند.
  دریافت پیام به معنی برقرار بودن اتصال (تا ۱۰ ثانیه دیگر) است و سرور باید برای اطلاع به کلاینت \متن‌لاتین{pong} را به همان کلاینت ارسال کند.

  \زیرقسمت{ذخیره کردن اطلاعات طرف مقابل}

  کلاینت‌های \متن‌لاتین{UDP} برای ارتباط با سرور تنها نیاز به پورت و آدرس دلال پیام دارند، اما سرور برای ارسال پیام به کلاینت‌ها
  می‌بایست از آخرین پورت و آی پی استفاده کند که با آن از کلاینت پیام دریافت کرده است. در نظر داشته باشید که برای این امر
  نیاز به دستور خاصی ندارید چرا که به صورت پیشفرض یک پورت تصادفی برای کلاینت شما انتخاب شده و می‌تواند از آن برای ارسال
  یا دریافت اطلاعات استفاده کند.

  \begin{latin}
  \begin{minted}[bgcolor=LightGray]{python}
import socket

client = socket.socket(socket.AF_INET, socket.SOCK_DGRAM) # UDP
while True:
    data, addr = client.recvfrom(1024)
    print("received message:",data,addr)
  \end{minted}
  \end{latin}

  در کد نمونه آورده شده سوکت \متن‌لاتین{UDP} به یک پورت آزاد و تصادفی مانند پورت ۳۷۰۲۰ بسته خواهد شد و به این ترتیب بسته‌های ارسال از پورت مبدا ۳۷۰۲۰ ارسال شده
  و از سوی دیگر می‌توان روی این پورت مطابق آنچه در مثال آورده شده است بسته دریافت نمود.

  \قسمت{‌مقایسه کنترل جریان در \متن‌لاتین{TCP} و \متن‌لاتین{UDP}}

  در قسمت قبلی با استفاده از کلاینت \متن‌لاتین{TCP} که از ارتباط خود چیزی دریافت نمی‌کرد بحث کنترل جریان
  در پروتکل \متن‌لاتین{TCP} را دیدید. در این قسمت قصد داریم همین موضوع را در پیاده‌سازی \متن‌لاتین{UDP} ببینیم.
  برای این موضوع، کلاینت \متن‌لاتین{UDP}ای دارید که روی یک موضوع مشترک شده است ولی بسته‌ای را \متن‌سیاه{نمی‌خواند}
  دقت داشته باشید که یعنی پس از ارسال و دریافت بسته‌های لازم جهت اشتراک بر یک موضوع خاص، دیگر بسته‌ای را دریافت نمی‌کند.
  یک کلاینت دیگر به صورت تکراری شروع به انتشار در همان موضوع می‌کند.
  با استفاده از نرم‌افزار \متن‌لاتین{Wireshark} توضیح دهید چه اتفاقی برای بسته‌های می‌افتد. آیا این بسته‌ها از دست می‌روند؟

  \فضای‌و*{\پر}
  \شروع{وسط‌چین}
این سند برپایه بسته \متن‌لاتین{\زی‌پرشین} گونه \متن‌لاتین{\گونه‌زی‌پرشین} توسعه پیدا کرده است.
  \پایان{وسط‌چین}

\end{document}
