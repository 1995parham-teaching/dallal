\documentclass[]{article}

\usepackage{hyperref}
\usepackage{import}
\usepackage{listings}
\usepackage{minted}
\usepackage[dvipsnames]{xcolor}
\usepackage{hyperref}
\usepackage{graphicx}
\usepackage{geometry}
 \geometry{
 a4paper,
 left=2.2cm,
 right=2.2cm,
 top=2.5cm,
 bottom=2.5cm
 }

\usepackage[localise]{xepersian}
\settextfont{Vazir}
\setlatintextfont{Roboto}

% for listing environments
\colorlet{LightGray}{Gray!30!}

% for href
\eqcommand{تارنما}{href}

\عنوان{دلال پیام - قسمت اول}
\نویسنده{پرهام الوانی}
\تاریخ{۲ آذر ۱۴۰۰}

\begin{document}
  \عنوان‌ساز
  \فهرست‌مطالب
  \صفحه‌شکن

  \قسمت{مقدمه}

  دلال پیام\پانویس{Message Broker} یک سرور است که پیام‌های دریافتی از کلاینت‌ها را بین آن‌ها تقسیم می‌کند. البته این دلال‌ها کارهای بیشتری هم انجام می‌دهند که خارج از بحث ما می‌باشد.
  دو عملیات مهم در سرور وجود دارد یکی \متن‌لاتین{Publish} کردن و دیگری \متن‌لاتین{Subscribe} کردن است. در عملیات \متن‌لاتین{Subscribe} کلاینت درخواست گوش دادن رو یک موضوع\پانویس{Topic} خاص را می‌دهد.
  در \متن‌لاتین{Publish} کلاینت یک پیام را روی یک موضوع خاص منتشر می‌کند. در نهایت دلال پیام‌ها را به دست \متن‌سیاه{تمام} کسانی که روی آن موضوع گوش می‌دهند می‌رساند.

  \قسمت{دستورات}

  کلاینت به سرور:

  \شروع{توضیح}
  \فقره[\متن‌لاتین{Publish}] \پررا \\ این دستور یک پیام از سمت کلاینت را تحت یک عنوان خاص برای سرور ارسال می‌کند. پیام‌ها رشته‌هایی با طول دلخواه می‌باشند.
  \فقره[\متن‌لاتین{‌Subscribe}] \پررا \\ این دستور به سرور اعلام می‌کند که این کلاینت متقاضی دریافت پیام‌های عنوان داده شده است.
  \فقره[\متن‌لاتین{Ping}] \پررا \\ این دستور از سمت کلاینت برای اطمنیان از ارتباط ارسال می‌گردد.
  \فقره[\متن‌لاتین{Pong}] \پررا \\ این دستور از سمت کلاینت در پاسخ به پیام \متن‌لاتین{Ping} ارسال می‌گردد.
  \پایان{توضیح}

  سرور به کلاینت:

  \شروع{توضیح}
  \فقره[\متن‌لاتین{Message}] \پررا \\ این دستور یک پیام از سمت سرور را برای کلاینت زمانی که متقاضی موضوعی است، ارسال می‌کند. پیام‌ها رشته‌هایی با طول دلخواه می‌باشند.
  \فقره[\متن‌لاتین{SubAck}] \پررا \\ این دستور از سمت سرور در جهت تایید پیام \متن‌لاتین{Subscribe} کلاینت زمانی که عملیات \متن‌لاتین{Subscribe} موفقیت آمیز باشد ارسال می‌شود.
  \فقره[\متن‌لاتین{PubAck}] \پررا \\ این دستور از سمت سرور در جهت تایید پیام \متن‌لاتین{Publish} کلاینت زمانی عملیات \متن‌لاتین{Publish} موفقیت آمیز باشد ارسال می‌شود.
  \فقره[\متن‌لاتین{Ping}] \پررا \\ این دستور از سمت سرور برای اطمینان از ارتباط ارسال می‌گردد.
  \فقره[\متن‌لاتین{Pong}] \پررا \\ این دستور از سمت سرور در پاسخ به پیام \متن‌لاتین{Ping} ارسال می‌گردد.
  \پایان{توضیح}


  \قسمت{پیاده‌سازی سرور}

  سرور یک سوکت سرور دارد که روی یک پورت مشخص گوش می‌دهد (در اینجا فرض می‌کنیم پورت موردنظر ۱۳۷۳ است). از آنجایی که قصد داریم سرور چند ارتباط همزمان را هندل کند نیاز به ساخت \متن‌لاتین{Thread} برای هر کانکشن خواهید داشت
  تا تابع \متن‌لاتین{accept} بلافاصله فراخوانی شود. سرور هیچ ارتباطی را نمی‌بیندد و همه ارتباط‌ها را به صورت باز نگه می‌دارد. برای سرور هر ارتباط یک کلاینت را نمایندگی می‌کند بنابراین می‌بایست لیست از سوکت‌ها و عنوانین موردنظر آن‌ها داشته باشد.

  \begin{latin}
  \begin{minted}[bgcolor=LightGray]{python}
import socket
import threading

HOST = '127.0.0.1'  # Standard loopback interface address (localhost)
PORT = 1373         # Port to listen on (non-privileged ports are > 1023)

def handler(conn, addr):
    with conn:
        print('Connected by', addr)
        while True:
            data = conn.recv(1024)
            if not data:
                break
            conn.sendall(data)
        print('Disconnected by', addr)

with socket.socket(socket.AF_INET, socket.SOCK_STREAM) as s:
    s.bind((HOST, PORT))
    s.listen()
    while True:
        conn, addr = s.accept()
        threading.Thread(target=handler, args=(conn, addr)).start()
  \end{minted}
  \end{latin}

  \قسمت{پیاده‌سازی کلاینت}

  کلاینت به صورت یک ابزار درون ترمینالی پیاده‌سازی می‌شود و نیازی به پیاده‌سازی گرافیکی \متن‌سیاه{نیست}.
  کلاینت آرگومان‌های زیر را در زمان اجرا دریافت می‌کند. برای اطلاعات بیشتر از آرگومان‌ها در پایتون می‌توانید از \تارنما{https://realpython.com/python-command-line-arguments/}{اینجا} استفاده کنید.
  کلاینت دو دستور اصلی دارد:

  \فضای‌و*{\پر}
  \شروع{وسط‌چین}
این سند برپایه بسته \متن‌لاتین{\زی‌پرشین} گونه \متن‌لاتین{\گونه‌زی‌پرشین} توسعه پیدا کرده است.
  \پایان{وسط‌چین}

\end{document}
